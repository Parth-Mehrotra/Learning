\def\llncs{0}       % 1 = using llncs



\documentclass[11pt]{article}
%\usepackage[margin=1.25in]{geometry}
%\usepackage[letterpaper,hmargin=1.25in,vmargin=1.25in]{geometry}
\usepackage{fullpage}
\usepackage{amsthm}

\usepackage{amsthm,amsmath,amsfonts,amssymb,amstext}
\usepackage{latexsym,ifthen,url,rotating}

%\usepackage[pdftex,usenames,dvipsnames]{color}
\usepackage[usenames,dvipsnames]{color}

\ifnum\llncs=0
\ifx\pdfoutput\undefined 

%% latex

\else

%% pdflatex

 \usepackage[pdftex,
    pdffitwindow,
    plainpages=false,
    pdfpagelabels=true,
    pdfborder={0 0 0},
    bookmarks=true,
    bookmarksnumbered=true,
    bookmarksopen=false
 ]{hyperref}

\fi
\fi

% --- -----------------------------------------------------------------
% --- Document-specific definitions.
% --- -----------------------------------------------------------------
\newtheorem{definition}{Definition}

\newcommand{\concat}{{\,\|\,}}
\newcommand{\bits}{\{0,1\}}

% --- -----------------------------------------------------------------
% --- The document starts here.
% --- -----------------------------------------------------------------
\begin{document}
%\maketitle
\sloppy

\noindent Rutgers University\\
CS206: Introduction to Discrete Structures II, Spring 2017\\
Professor David Cash\\

\begin{center}
\LARGE{\textbf{Homework 5}}\\
\large{\textbf{\emph{Due at 9am on Tuesday, Feb 28}}}
\end{center}

\vspace{.1in}

\noindent\textbf{Instructions:} \begin{itemize}

\item Follow the collaboration policy stated in class:  You may discuss
problems together \emph{but you must write up your own solutions}.  You should
never see your collaborators' papers.  If we detect copying, it may be reported
to the administration.

\item Submitting homework:  You may bring your written homework to class.  You
may instead opt to turn your homework in via Sakai, but we will only accept
\emph{typed} solutions, as PDFs created using a program like MS Word (using
print to pdf), or ideally, Latex\footnote{See, for
example~\url{https://www.latex-project.org/get/}, and numerous tutorials
online.  Latex is the standard language for producing nice-looking documents
with lots of math (including homework and exams in this class).}.  \textbf{Let
me repeat: If submitting via Sakai, you must type up your solutions as a PDF
file.  Scans and images of handwritten documents will not be accepted.}

\item Point values for each problem are listed.  Some homework assignments will
be worth more points than others.  

\item Homework grades are not curved.  You should aim to earn all of the points
on each homework, and I encourage you seek out help via office hours,
recitation, and the study group.

\end{itemize}



\noindent\textbf{Graded Problems:}


\begin{enumerate}

\item 
Give a story proof that
\[
\sum_{k=1}^n k \binom{n}{k}\binom{n}{n-k} = n \binom{2n -1}{n-1}
\]
for all positive integers $n$. \textbf{(10 pts)}

\emph{Hint:  Consider choosing a committee of size $n$ and a president (who is on the committee) from a
group of $n$ freshman and $n$ sophomores, where only freshman are eligible to
be president.  Argue that both sides of the equation count this, starting
with the right hand side which is easier.}

\item Problem 14 in section 1.2.5. \textbf{(6 pts)}

\item Problem 26 in section 1.2.5. \textbf{(6 pts)}

\item Suppose we deal a 5 card hand from a standard deck.  What is the
probability that the hand is all Spades, given that it has at least two
Spades? \textbf{(6 pts)}

\item The Jack of Spades (with cider), Jack of Hearts (with tarts), 
Queen of Spades (with a wink), and Queen of Hearts (without tarts) are
taken from a deck of cards.  These four cards are shuffled, and then two are
dealt.

\begin{enumerate}
\item Find the probability that both of these two cards are queens, given
that the first card dealt is a queen. \textbf{(5 pts)}

\item Find the probability that both of these two cards are queens, given
that at least one is a queen. \textbf{(5 pts)}

\item Find the probability that both of these two cards are queens, given
that one is Queen of Hearts. \textbf{(5 pts)}

\end{enumerate}

\item Let $A$ and $B$ be events with $0 < P(A \cap B) < P(A) < P(B) < P(A \cup
B) < 1$.  You are hoping that \emph{both} $A$ and $B$ occurred.  Which of the
following piece of information would you be happiest to observe:  That $A$
occurred, that $B$ occurred, or that $A\cup B$ occurred? \textbf{(6 pts)}


\item According to the CDC (Centers for Disease Control and Prevention),
men who smoke are 23 times more likely to develop lunch cancer than men
who don't smoke.  Also according to the CDC, 21.6\% of men in the U.S. smoke.
What is the probability that a man in the U.S. is a smoker, given that he
develops lung cancer? \textbf{(6 pts)}

\item A hat contains 100 coins, where 99 are fair but one is double-headed
(always landing Heads).  A coin is chosen uniformly at random.  The chosen coin
is flipped 7 times, and it lands Heads all 7 times.  Given this information,
what is the probability that the chosen coin is double-headed? \textbf{(6 pts)}



\item \textbf{Extra credit.} In this problem we will find a proof that
\[
\binom{\binom{n}{2}}{2} = 3\binom{n+1}{4}.
\]
This is broken into sub-problems below.
\begin{enumerate}

\item  How many ways are there to select $4$ people from
a group of $n$, and then break those $4$ people into
two teams of two?  (The order of the teams does not matter.) \textbf{(3 pts)}

\item How many ways are there to select $3$ people from a group of
$n$, and then from those $3$ people, create two teams of two that have exactly
one
overlapping member?  (Here the order of those two teams does not matter, like
the first part.) \textbf{(3 pts)}

\item By simplifying and applying Pascal's relation argue that the
sum of your answers from the first two parts equals $3\binom{n+1}{4}$.
\textbf{(3 pts)}

\item Give a story proof that $\binom{\binom{n}{2}}{2}$ is also
equal to the sum of your answers from the first two parts.  This will complete
the proof of the identity.
\textbf{(3 pts)}

\emph{Hint:  Observe that $\binom{n}{2}$ is the number of ways we can form a
group of two from amongst $n$ possible people.  $\binom{\binom{n}{2}}{2}$ is
the number of ways we can pick two different groups of two.}

\end{enumerate}



\end{enumerate}



\end{document}

